\chapter[METODOLOGI PENELITIAN]{\\ METODOLOGI PENELITIAN}

\section{Perangkat Penelitian}
\section{Tahapan Penelitian}
\subsection{Studi Literatur}
Bagian ini digunakan apabila dibutuhkan, silahkan bisa ditambah atau dikurangi sesuai kebutuhan.
\subsection{Perancangan Sistem}
Bagian ini digunakan apabila dibutuhkan, silahkan bisa ditambah atau dikurangi sesuai kebutuhan.
\subsection{Pengujian}
Bagian ini digunakan apabila dibutuhkan, silahkan bisa ditambah atau dikurangi sesuai kebutuhan.
\subsection{\textit{Pre-Processing Data dan Metode ...}}
Bagian ini digunakan apabila dibutuhkan, silahkan bisa ditambah atau dikurangi sesuai kebutuhan.
\subsection{Pembuatan Laporan}
Bagian ini digunakan apabila dibutuhkan, silahkan bisa ditambah atau dikurangi sesuai kebutuhan.
\section{Perancangan Sistem dan Implementasi}
Bagian ini digunakan apabila dibutuhkan, silahkan bisa ditambah atau dikurangi sesuai kebutuhan.
\subsection{Blok Diagram}
Bagian ini digunakan apabila dibutuhkan, silahkan bisa ditambah atau dikurangi sesuai kebutuhan.
\subsection{Perancangan Perangkat Lunak \textit{Software}}
Bagian ini digunakan apabila dibutuhkan, silahkan bisa ditambah atau dikurangi sesuai kebutuhan.
\subsubsection{Flowchart}
\subsubsection{Pengambilan Dataset}
\subsubsection{Training Dataset}
\subsubsection{Testing Hasil Training Dataset}
Bagian ini digunakan apabila dibutuhkan, silahkan bisa ditambah atau dikurangi sesuai kebutuhan.
\section{Metode Pengambilan Data}
\section{Metode Analisis Data}
