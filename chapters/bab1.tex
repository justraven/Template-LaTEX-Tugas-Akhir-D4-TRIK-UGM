\chapter[PENDAHULUAN]{\\ PENDAHULUAN}

\section{Latar Belakang}

\section{Rumusan Masalah}

\section{Batasan Masalah}

\section{Tujuan Penelitian}


\section{Manfaat Penelitian}



\section{Sistematika Penulisan}
Sistematika penulisan adalah susunan atau struktur dari laporan proyek akhir sarjana terapan yang menjabarkan bagian-bagian yang harus ada dalam laporan proyek akhir sarjana terapan. Sistematika penulisan dapat berbeda antara satu institusi dengan institusi lainnya, namun umumnya terdiri dari beberapa bagian yang wajib ada, seperti :

\begin{packed_item}
    \item Bab 1 Pendahuluan
    \item Bab 2 Tinjauan Pustaka
    \item Bab 3 Desain dan Implementasi
    \item Bab 4 Hasil dan Pembahasan
    \item Bab 5 Kesimpulan dan Saran
    \item Daftar Pustaka
\end{packed_item}

Penjelasan detail dari masing-masing bab adalah sebagai berikut:

\begin{packed_enum}
    \item Bab 1 Pendahuluan : menjelaskan latar belakang, rumusan masalah, tujuan, batasan masalah, serta sistematika penulisan dari laporan proyek akhir sarjana terapan.
    \item Bab 2 Tinjauan Pustaka : menjabarkan tentang studi yang telah dilakukan oleh peneliti sebelumnya yang berhubungan dengan topik yang diteliti dalam proyek akhir sarjana terapan, serta membahas teori yang relevan dengan masalah yang akan diteliti. Bab ini berisi tentang kajian pustaka yang diperoleh dari berbagai sumber yang terkait dengan masalah yang akan diteliti.
    \item Bab 3 Desain dan Implementasi: menjabarkan tentang rencana dan perencanaan yang digunakan dalam melakukan penelitian dan pelaksanaan penelitian sesuai dengan rencana yang telah ditetapkan dalam desain penelitian. Desain penelitian terdiri dari beberapa elemen, seperti desain penelitian, metode pengumpulan data, sampel, dan analisis data. Implementasi meliputi tahap-tahap dari pelaksanaan penelitian, seperti pengambilan sampel, pengumpulan data, dan analisis data.
    \item Bab 4 Hasil dan Pembahasan: menjabarkan hasil yang diperoleh dari proyek akhir sarjana terapan dan memberikan pembahasan yang mendalam terkait dengan hasil tersebut. Bab ini juga berisi tentang interpretasi data yang diperoleh dari penelitian.
    \item Bab 5 Kesimpulan dan Saran: menjabarkan kesimpulan yang diperoleh dari proyek akhir sarjana terapan serta saran yang diberikan untuk penelitian selanjutnya.
    \item Daftar Pustaka : menjabarkan sumber-sumber yang digunakan dalam laporan proyek akhir sarjana terapan.
\end{packed_enum}

Secara keseluruhan, sistematika penulisan dalam laporan proyek akhir sarjana terapan adalah susunan atau struktur dari laporan proyek akhir sarjana terapan yang menjabarkan bagian-bagian yang harus ada dalam laporan proyek akhir sarjana terapan, yang meliputi Pendahuluan, Tinjauan Pustaka, Metode Penelitian, Hasil dan Pembahasan, Kesimpulan dan Saran, serta Daftar Pustaka. Sistematika penulisan yang baik akan membuat laporan proyek akhir sarjana terapan lebih mudah untuk dibaca dan dipahami.